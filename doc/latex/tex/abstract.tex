RISC-V is a new instruction set architecture developed at the University of California, Berkley. Its main differential and what has made this architecture promising is that it is an Open Source ISA. This project proposes an environment for RISC-V architecture assembly code development. This environment is aimed at learning, from the code written in a text editor in the browser, the user can assemble and simulate the code and then visualize various results of the written code. This system does not require installations because it works on a server accessible through the internet, facilitating the beginning of learning the language and architecture that are the main objectives of the system. We can see through example codes, such as the Fibonacci sequence, register values, memory, assembled code, color map representing a section of memory. The simulation takes place in three modes, step by step automatic, step by step manual, or instantaneously. For the future other modules can be implemented, extended to 64 bits, and also reduced set of instructions. Usability features can also be improved, for example being able to save, download, upload codes.