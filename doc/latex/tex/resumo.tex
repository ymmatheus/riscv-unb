RISC-V é uma nova arquitetura de conjunto de instruções desenvolvida na Univerisdade da Califórnia, Berkley. Seu principal diferencial e o que têm tornado esta arquitetura promissora é o fato de ser uma ISA Open-Source. Este projeto propõe um ambiente para desenvolvimento de código Assembly da arquitetura RISC-V. Este ambiente é voltado para o aprendizado podendo, a partir do código escrito em um editor de texto no browser, montar e simular o código e então visualizar vários resultados do código escrito. Este sistema não necessita de instalações pois funciona em um servidor acessível pela internet, facilitando o início da aprendizagem da linguagem e arquitetura que são os objetivos principais do sistema. Podemos ver atráves de códigos exemplos, como a sequência de Fibonacci, valores de registradores, memória, código montado, mapa de cores representando uma seção da memória. A simulação ocorre de três maneiras, passo a passo automático, passo a passo manual, ou instantaneamente. Para o futuro outros módulos podem ser implementados, extender para 64 bits, e também conjunto de instruções reduzidas. Questões de usabilidade também podem ser melhoradas, por exemplo, ser capaz de salvar, baixar, fazer upload de códigos.