
%%%%%%%%%%%%%%%%%%%%%%%%%%%%%%%%%%%%%%%%%%%%%%%%%%%%%%%%%%%%%%%%%%%%%%%%%%%%%%%%
%%%%%%%%%%%%%%%%%%%%%%%%%%%%%%%%%%%%%%%%%%%%%%%%%%%%%%%%%%%%%%%%%%%%%%%%%%%%%%%%
%%%%%%%%%%%%%%%%%%%%%%%%%%%%%%%%%%%%%%%%%%%%%%%%%%%%%%%%%%%%%%%%%%%%%%%%%%%%%%%%
\section{RISC-V}%

	RISC-V é uma arquitetura de conjunto de instruções aberta, criada na Universidade da Califórnia, em Berkeley. Originalmente foi pensada para ser utilizada na pesquisa e ensino da área de arquitetura de computadores, mas está se tornando um padrão de arquitetura aberta para a indústria.~\cite{riscv_spec}

	Sua arquitetura obedece aos padrões RISC (Reduced Instruction Set Computing), tendo instruções simples e completas. Foi projetada para ser rápida, ocupar pouco espaço físico, ter baixo consumo de energia, ser extensível, e compatível com entre suas versões.

	Seu nome é pronunciado na língua inglesa como \textit{"risc five"}. O motivo de ser \textit{"five"} é devido ao fato de que é o quinto maior projeto de uma ISA RISC desenvolvida na UC Berkeley. As primeiras foram RISC-I, RISC-II, SOAR, e SPUR. O numeral romano "V" de RISC-V também funciona com significado de \textit{"variations"} e \textit{"vectors"}.

	\subsection{Objetivos}

		Seus projetistas sempre são perguntados o motivo ao qual eles quiseram desenvolver uma nova ISA. Alguns dos motivos para o qual usar uma ISA comercial são a existência de suporte de um ecossistema de software, tanto ferramentas de desenvolvimento, portabilidade e ferramentas educacionais, outros benefícios seriam a grande quantidade de documentação, tutoriais e exemplos para o desenvolvimento.\\

		Porém estas vantagens são pequenas na prática, e listam várias disvantagens ao utilizar ISAs comerciais,

		\begin{itemize}
			\item ISAs comerciais são proprietárias
			\item ISAs comerciais são populares somente em alguns nichos do mercado
			\item ISAs comerciais vem e vão
			\item ISAs populares são complexas
			\item ISAs comerciais dependem de outros fatores para trazer aplicações
			\item ISAs comerciais populares não são projetadas para extensibilidade
			\item Uma ISA comercial modificada é uma nova ISA
		\end{itemize}

		A posição dos projetistas é que, em um sistema computacional, a ISA talvez seja a interface mais importante, e não existe razão pra que esta seja proprietária.

		

			- eficiente energetica\\
			- compatibilidade\\
			- simples\\
			- escalavel\\
			- modular\\




	\subsection{História}%
		A ISA RISC-V foi originalmente desenvolvida na Universidade da Califórnia, Berkeley, na Divisão de Ciência da computação, no departamento de Engenharia Elétrica e Ciência da Computação. Baseada na experiência com projetos passados de seus projetistas, a definição da ISA foi iniciada no verão de 2010.\\

		Os primeiros processadores RISC-V fabricados foram escritos em Verilog e manufaturados em tecnologia de pré-produção de 28 nm FD-SOI (\textit{Fully Depleted Silicon On Insulator}) da companhia STMicroeletronics com o nome \textit{Raven-1}

		%Andrew Waterman and Yunsup Lee developed the C++ ISA simulator “Spike”, used as a golden model in development and named after the golden spike used to celebrate completion of the US transcontinental railway. Spike has been made available as a BSD open-source project.

	

	\subsection{RISC-V Foundation}
		
		A \textit{RISC-V Foundation} é uma organização sem fins lucrativos, criada para direcionar futuro desenvolvimento e incentivar a utilização da ISA RISC-V.\ ~\cite{riscv_foundation} 

		O presidente do conselho atualmente é Krste Asanovic, professor do departamento de Engenharia elétrica e ciência da computação na Univerisdade da Califórnia em Berkeley. Também co-fundado da empresa SiFive Inc., a qual incentiva do uso comercial de processadores RISC-V.\

		E o vice-presidente é o professor David Patterson, muito conhecido pelo livro \textit{Computer Architecture: A Quantitative Approach}, que escreveu juntamente com John Hennessy, e suas pesquisas relacionadas a RISC, RAID, e Redes de estações de trabalho.\

		Outros membros incluem:\
		\begin{itemize}  
			\item Zvonimir Bandic, pesquisador e diretor da Western Digital Corporation. 
			\item Charlie Hauck, CEO da Bluespec Inc.
			\item Frans Sijstermans, vice presidente de engenharia da NVIDIA.
			\item Ted Speers, chefe de arquitetura de produtos e planejamento do grupo SoC da Microsemi.
			\item Rob Oshana, gerente de desenvolvimento de software e negócios em segurança na NXP Semiconductors. 
		\end{itemize}

		e também, Sue Leininger, gerente de comunidade e Rick O’Connor, director executivo.~\cite{riscv_fboardmembers}

	\subsection{Open Source}
		
		O fato da arquitetura ser Open Source traz inúmeras vantagens, principalmente na parte de distribuição e compartilhamento.
		vantagens\\
			vulnerabilidades\\
			the spectre and meltdown vulnerabilities\\
		licensa\\




\section{Objetivos e motivação deste projeto}

	RISC-V tem ganhado muito espaço tanto na academia quanto na industria, começar um estudo desta arquitetura na universidade de brasilia e prover ferramentas para facilitar o inicio destes estudos.

	ao iniciar o projeto havia poucas ferramentas e as ferramentas que haviam eram de dificil instalação configuração.

	este projeto é um primeiro estudo da arquitetura em questao, tem o intuito de ser de facil utilização por isso foi feito para ser utilizada na web, e ser extensivel e facil legibildiade, por isso a escolha da linguagem python, por ser mais acessível,

	para casos onde se deseja performance, podem conectados modulos através de extensões para python como cython~\cite{cython_home}



importancia do riscv\\
			- academica\\
			- industrial\\



\section{Processadores MIPS, ARM e RISC-V}%

		- historia entre eles
		- comparacoes
