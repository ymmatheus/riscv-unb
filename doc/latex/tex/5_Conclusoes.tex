\section{Objetivos atingidos}

	Neste projeto conseguimos realizar um primeiro estudo da arquitetura RISC-V, e implementar uma interface de fácil utilização, multiplataforma, onde um usuário pode escrever códigos na linguagem assembly RISC-V, realizar a montagem do código, e rodar uma simulação deste código.

	O sistema permite exportar as saídas do montador para que o usuário possa simular seu código em outras ferramentas de simulação, ou uma implementação própria em uma \textit{FPGA}.

	Os módulos de montagem e simulação foram implementados como biblitoecas separadas, por isso podem ser utilizadas em outros contextos. Por exemplo, em linha de comando.

\section{Pontos positivos e negativos}

	Alguns pontos positivos da ferramenta são aqueles já citados quando descrevemos as vantagens de se utilizar uma plataforma web, e alguns outros,
	\begin{itemize}
		\item Não necessidade de instalação e configurações para começar a utilzar a ferramenta
		\item Multiplataforma
		\item Extensibilidade
	\end{itemize}

	Desvantagens dessa solução hospedada são,

	\begin{itemize}
		\item Custo de hospedagem
		\item Possível indisponibilidade	
	\end{itemize}

	Porém, apesar dessas desvantagens citadas acima, o fato da solução ser aberta diminui o peso dessas desvantagens, pois o código pode ser baixado e rodado localmente. 

	Assim, mesmo que haja algum problema de hospedagem, pode-se obter facilmente o código fonte, e tendo um interpretador instalado em seu computador, se pode fazer sua própria hospedagem local, pois o interpretador já fornece um servidor web imbutido nativamente para desenvolvimento.


\section{Dificuldades}

	No começo do projeto tentou-se implementar um editor de texto em javascript sem utilização de bibliotecas terceiras, porém a tarefa se mostrou muito difícil de ser realizada, então optou-se por utilizar a biblioteca CodeMirror, muito utilizada por vários sites populares de desenvolvimento.

	Na parte do montador foi onde houve o maior esforço. A parte documentada que diz respeito a linguagem de programação ainda é escassa, porém para a solução limitada que foi feita neste projeto foi o suficiente. Por ser o módulo que interage diretamente com a entrada do usuário, este se mostrou ser o ponto crítico da solução.

	O simulador foi bem simplificado, por isso não houveram grandes dificuldades. O desenvolvimento do simulador tinham algumas dificuldades de saber como as instruções por exemplo de "branch" funcionavam à nível de hardware, pelo tipo de codificação utilizada, pois algumas informações sobre estas não ficam explicitas nas documentações.

	Grande parte do esforço também foi direcionado à integração do frontend com o backend, pois pode-se considerar dois sistemas diferentes. O que era um dos objetivos, para que a ferramenta seja flexível, e possa ser extendida e evoluida mais a frente. Por isso foi decidido utilizar uma API para que seja flexível utilizar novas interfaces ou novos montadores, simuladores customizados.

\section{Melhorias e trabalhos futuros}

	Existem muitas funcionalidades a serem feitas, citando algumas,

	\begin{itemize}
		\item Expandir arquitetura para 64-bits
		\item Sistema de login/logout, com um sistema de usuários, poderiamos montar um sistema para salvar e compartilhar códigos com outros usuários. Também inserir módulos personalizados.
		\item Contador de instruções.
		\item Bitmap mapping, para termos uma representação visual da memória do sistema.
		\item Adição de instruções e pseudo-instruções customizadas.
		\item Utilização de bibliotecas ou frameworks mais modernos para frontend como ReactJS, AngularJS, VueJS.
		\item Melhorar interface para dispositivos móveis.
		\item Step run. Rodar uma instrução por vez.
	\end{itemize}

	A parte de extensões da solução não foi realizada, porém teria muitas aplicações e diferenciaria bastante de outras ferramentas. Para isso precisaríamos ter uma documentação bem especificada, para que haja compatibilidade da interface entre os módulos de extensão.

	Uma das funcionalidades que seria muito interessante ter realizado no projeto seria a possível extensão de códigos em C, para podermos utilizar modelos descritos em mais baixo nível feitos por exemplo com a biblioteca \textit{SystemC}. Com isso poderíamos testar com maior facilidade implementações baseados em performace.

