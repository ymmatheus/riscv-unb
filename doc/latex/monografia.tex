%%%%%%%%%%%%%%%%%%%%%%%%%%%%%%%%%%%%%%%%
% Classe do documento
%%%%%%%%%%%%%%%%%%%%%%%%%%%%%%%%%%%%%%%%

% Opções:
%  - Graduação: bacharelado|engenharia|licenciatura
%  - Pós-graduação: [qualificacao], mestrado|doutorado, ppca|ppginf

% \documentclass[engenharia]{UnB-CIC}%
\documentclass[engenharia]{UnB-CIC}%

\usepackage{pdfpages}% incluir PDFs, usado no apêndice


%%%%%%%%%%%%%%%%%%%%%%%%%%%%%%%%%%%%%%%%
% Informações do Trabalho
%%%%%%%%%%%%%%%%%%%%%%%%%%%%%%%%%%%%%%%%
\orientador{\prof \dr Ricardo Pezzuol Jacobi}{CIC/UnB}%
%\coorientador{\prof \dr José Ralha}{CIC/UnB}
\coordenador{\prof \dr Ricardo Pezzuol Jacobi}{CIC/UnB}%
\diamesano{26}{março}{2017}%

\membrobanca{\prof \dr Donald Knuth}{Stanford University}%
\membrobanca{\dr Leslie Lamport}{Microsoft Research}%

\autor{Matheus Y.}{Matsumoto}%

\titulo{RISC-V: Ambiente de montagem e simulação}%

\palavraschave{risc,LaTeX, metodologia científica, trabalho de conclusão de curso}%
\keywords{LaTeX, scientific method, thesis}%

\newcommand{\unbcic}{\texttt{UnB-CIC}}%

%%%%%%%%%%%%%%%%%%%%%%%%%%%%%%%%%%%%%%%%
% Texto
%%%%%%%%%%%%%%%%%%%%%%%%%%%%%%%%%%%%%%%%
\begin{document}%
    \capitulo{1_Introducao}{Introdução}%
    \capitulo{2_Fundamentacao}{Fundamentação teórica}%
    \capitulo{3_Ambiente}{Ambiente Proposto}%
    \capitulo{4_Resultados}{Resultados e Avaliação do Sistema}%
    \capitulo{5_Conclusoes}{Conclusões}%
    %\capitulo{2_UnB-CIC}{A Classe \unbcic}%
    %\capitulo{3_TCC}{Trabalho de Conclusão de Curso}%
    %\capitulo{4_Apresentacao}{Apresentações}%

    %\apendice{Apendice_Fichamento}{Fichamento de Artigo Científico}%
    %\anexo{Anexo1}{Documentação Original \unbcic\ (parcial)}%
\end{document}%